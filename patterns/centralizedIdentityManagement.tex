\newpage
\section*{CENTRALIZED IDENTITY MANAGEMENT}
also known as: IDENTITY MANAGEMENT BUS.

The system makes use of user identities which need to be managed. 

\begin{center}
\ding{118} \ding{118} \ding{118}
\end{center}

\textbf{Decentralized user identity management means a lot of extra work as identities have to be managed on many different places and it is hard to get a centralized overview of all existing or available identities. This also makes role management much more complex.}\\

\textit{Separation of Duties.} Especially when Separation of Duties (SoD) is a concern such as within financial environments it is important for organizations to be able to show to e.g. an EDP auditor that all regulations are fulfilled.

\begin{center}
\ding{118} \ding{118} \ding{118} 
\end{center}

\textbf{Therefore: Make use of a centralized identity management system if this is available.}\\

This solution has several advantages: if the centralized identity management system (CIM) is also connected to the human resources system (HRM)-system, it is easier to revoke certain grants due to retirements etc. Also user roles could be (automatically) inferred from function profiles in the HRM system.

%Relation with possible {\sc Centralized Role Management} pattern and {\sc Role Based Access Control (RBAC)} and at a lower level {\sc Access Control List (ACL)}?

When no centralized identity management system is available a lot of organizations make use of something like Active Directory Services (ADS). Mostly in these cases where ADS is used this isn't connected to a HR system whereby the events or triggers for the HR processes placing in, leave service, function change or department change are missed in the ADS. This causes an increase in maintenance activities to take care of pollution of the ADS.

%TODO: describe an alternative solution if no centralized identity management system is available (RB: dit is juist het probleem wat hiermee opgelost dient te worden. Heel veel organisaties gebruiken dan alleen het ADS (Active Directory Services) als identity source, maar dat is meestal niet gekoppeld aan het personeelssysteem, waardoor de events/triggers voor bv. in-/uitdienst, functiewijziging of afdelingswijziging gemist worden en de situatie vanuit beheeroogpunt lastig beheerbaar wordt).

\begin{center}
\ding{118} \ding{118} \ding{118} 
\end{center}

There is an urgent need within medium to large organizations to centralize role based access information. Several applications have role based access information. This dispersion of information leads to a high maintenance sensitivity. Which demands a high level of deployment. Therefore the dispersion of information needs to be centralized within a solution according to this pattern.
% Many organisations have several identity sources, e.g. HR-system, Active Directory, access management systems (physical). --> mochten we ook metadirectories willen toelichten, maar voor nu laat ik het even buiten beschouwing.