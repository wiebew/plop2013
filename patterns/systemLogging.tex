\newpage
\section*{USE BUILT-IN SYSTEM LOGGING}

The application needs to provide the ability of logging certain events or actions by using the built in system logging of a platform. 

\begin{center}
\ding{118} \ding{118} \ding{118}
\end{center}

\textbf{Having a variety of logging formats and log-file locations makes it hard to monitor the state of a whole enterprise, including all running applications.}\\

\textit{Format Variety.} A high variety of logging formats increases the complexity of integrating the information held within those several log files. It becomes a burden to nullify the different lay-outs of these log files.\\ 

\textit{Location Variety.} When having a variety of log file locations the dispersion of those locations makes it difficult to gather those files to one stack.

\begin{center}
\ding{118} \ding{118} \ding{118} 
\end{center}

\textbf{Therefore: Use the built-in system logging mechanism or define a standard format to be used by all systems.}\\

Don't reinvent the wheel. Many monitoring tools use the system built-in logging mechanisms so don't try to circumvent it. This allows the administrators to make use of existing tools that collect, centralize, and search the logs \cite{Limoncelli2011a}.

If it is not possible to use the built-in system logging, e.g. because of different operating systems being used, then define a standard for your system landscape and ensure that this is used for logging. Combine this approach with {\sc Single File Location}.

\begin{center}
\ding{118} \ding{118} \ding{118} 
\end{center}

Besides the above mentioned reasons, it is a lot easier to automatically generate incidents from specific defined events from the built-in system log for an IT service management (ITSM) tool. This ITSM tool can be configured to forward the automatically generated incidents directly, without human intervention, to the second line specialists.