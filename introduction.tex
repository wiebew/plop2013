\section{Introduction} 

%Unknown1: precise, concise & clear
%U1: Weak description of the context background and ...
%Leo: 
% - Problem: Is touched, but not explained.
%            What are these 10 do's & don'ts?
%            What is the relationship to 3 patterns?
% - Contribution: Weak: - What is the problem?
%                       - Is it not solved before?
% - Portability is covered by layers, adapters, ...
% - Missing: - Scope, Research Question
%            - Method
%            - Paper intro
In the article \textit{A plea from sysadmins to software vendors: 10 Do's and Don'ts} by Thomas Limoncelli \cite{Limoncelli2011a}, system administrators collected a basic list of do's and dont's (guidelines) for software vendors in order to make the life of the system administrators more easy. For fully completeness we are citing this list:
\begin{itemize}
	\item DO have a "silent install" option.
	\item DON'T make the administrative interface a GUI.
	\item DO create an API so the system can be remotely administered.
	\item DO have a configuration file that is an ASCII file, not a binary blob.
	\item DO include a clearly defined method to restore all user data, a single user's data, and individual items (for example, one email message).
	\item DO instrument the system so we can monitor more than just, "Is it up or down?"
	\item DO tell us about security issues.
	\item DO use the built-in system logging mechanism (Unix syslog or
Windows Event Logs).
	\item DON'T scribble all over the disk.
	\item DO publish documentation electronically on your Web
site.
\end{itemize}

For a large number of points in this list there is a high agreement between administrators on what the best practices should be. However as system adminstrators are on the "receiving" end for a new or modified application, it is necessary to influence other parties who have a key position in creating or changing an application. 

A role that fits this key position is the software architect. Among other concerns the software architect is responsible for the software architecture. The software architecture is the main design document for the software of an application. The design decisions taken in that document have a profound impact on the workload of the system administrators. 

This paper aims at influencing the software architects and the software architecture by providing patterns for software architecture that are endorsed by system administrators.

%U1: functional & non-functional aspects
The focus of Software Architecture is often on realizing quality attributes, such as those described in ISO 25010: functional suitability, performance efficiency, compatibility, usability, reliability, security, maintainability and portability. Many patterns have been described, e.g. in the POSA book series \cite{Buschmann1996}, and their general applicability for realizing the qualities has been discussed \cite{Harrison2011}. 
%Peter: What is a System Administrator?
%- What does he do?
%Examples? Upfront
%Unknown1: I agree that administration aspects are close to portability. Why the reference to security for example?
%Unknown1: adaptability is vague in this way!!
There are also publications that focus on patterns for specific quality aspects, like patterns for fault tolerant systems \cite{Hanmer2007} or security patterns \cite{Schumacher2005}.
%Rebecca: This (But there is one quality attribute where ...) is the heart of your reasons for writing these patterns, but it is burried here. Should be upfront. Intro is way too long. (Roland / Christian)
%Peter: claim about Portability is unproven? Relevant
 But there is one quality attribute where not much attention has been paid to: Portability and its sub-qualities Adaptability, Installability, and Replaceability. A number of concerns from system administrators are covered by the aforementioned attributes, but the mapping of the concerns on the attributes is not intuitive.

Both approaches --- software architecture patterns for realizing the above described quality attributes and patterns that support the work of system administrators --- don't touch some important aspects of the intersection of software architecture and system administration. Therefore we want to introduce a set of patterns which bridges this gap, based on the needs of the system administrators. 

The problems that are cited in the aforementioned article have been experienced within daily system administration practice. 

%Eltjo Poort proposes to view Software Architecture as a risk- and cost management discipline.  An architect should make decisions that %have the highest impact in terms of risk and cost \cite{Poort2011}. Software Architects in projects tend to view risk and costs in %relation to project delivery.  As a substantial amount of applications costs are incurred after the project phase, it would be %beneficial to include approaches in the software architecture to reduce risks and costs for application management. 

With these patterns we want to give ideas to software architects and application developers on how to improve their applications from a system administration viewpoint. 
%Unknown1: Audience?? ..., Admin, Application development?
%\subsection*{Further Ideas} 
%\begin{itemize}
%	\item Hergebruik Zorg voor een standaardbibliotheek van standaard gebruikte oplossingen. Bijv. Logging, Security, Datumafhandeling, format van veelgebruikte user-data. Beheerders ondervinden veel last van dat voor iedere applicatie dezelfde functionaliteit toch net even anders gerealiseerd is en anders geconfigureerd. \textit{Comment Christian: dit zou in een pattern gaan richting {\sc Standardized Administration Tools} ofzo, dit zou dan wel {\sc Use Built-in System Logging} kunnen vervangen of specifieker maken.} 

%	\item Goede functionele beschrijving. Redelijke beschrijving van wat een systeem globaal doet, uitgebreid met technische details rondom de werking. Waar draait de applicatie, waarom is deze applicatie nodig? Welke interfaces zijn er, wie gebruikt deze interfaces? Configuratie instellingen. \textit{Comment Christian: dit staat ook in het CACM artikel. Maar waarom is dit belangrijk, dus wat is het probleem als je het niet doet?; RB: het belang van (de juiste hoeveelheid) documentatie wordt al op heel veel plekken in de literatuur genoemd.} 
%	\item NFR`s Hoeveel verkeer, hoe lang mag een transactie duren? Welke patterns worden gebruikt: synchroon/asynchroon. Heldere QoS/SLA: Beschikbaarheid, support, onderhoudsvensters, audit en logging. \textit{Comment Christian: Hier weet ik nog niet in hoeverre dat binnen de scope van dit paper valt. Kunnen we morgen bespreken.; RB: mee eens, vandaar --> commented} 
%\end{itemize}
