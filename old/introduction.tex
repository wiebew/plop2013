\section{Introduction} These patterns are inspired by the article \textit{A plea from sysadmins to software vendors: 10 Do's and Don'ts} by Thomas Limoncelli, published in the Communications of the ACM magazine \cite{Limoncelli2011a}. Refer to POSA \cite{Buschmann1996}. 

Eltjo Poort proposes to view Software Architecture as a risk- and cost management discipline.  An architect should make decisions that have the highest impact in terms of risk and cost \cite{Poort2011}. Software Architects in projects tend to view risk and costs in relation to project delivery.  As a substantial amount of applications costs are incurred after the project phase, it would be beneficial to include the software architecture with approaches to reduce risks and costs for application management. 

Focus of Software Architecture is often on realizing quality attributes, such as those described in ISO 25001: Functional suitability, Performance efficiency, Compatibility, Usability, Reliability, Security, Maintainability and Portability. Many patterns have been described and their applicability for realizing the qualities has been discussed (\textbf{TODO}: cite e.g. Harrison/Avgeriou for general (WW, heb geen mooie publicatie van Paris die hier lekker op past kunnen vinden), also Hanmer for fault tolerant systems \cite{Hanmer2007}, Eduardo Fernandez for Security Patterns, look for more ISO-related ones). But there is one quality attribute where not much attention has been paid to: Portability and its sub-qualities Adaptability, Installability, and Replaceability. \textbf{TODO}: make this list more precise (WW: removed conformance => not part of ISO25010)).

There have been several initiatives to describe infrastructure and middleware patterns from the perspective of a system administrator: 
\begin{itemize}
	\item Daniel Jumelet: Open Infrastructure Architecture repository (OIAr)\footnote{\url{http://www.infra-repository.org/oiar/index.php/Main_Page}} - this site provides a wide variety of infrastructure patterns for several working areas: Client Realm, Middleware, Network, Security + Support, Server, Storage. Beside this repository also contains architecture \& design guidelines in the form of construction models at various levels and from various angles. It is constructed by making use of one of the most important tools of OIAm: The Building Blocks Model. The Building Blocks Model is primarily a \textit{decomposition} tool. That means that it is used to dissect infrastructure landscapes into logical dimensions and parts in order to enable structured and methodological modeling (\textit{composition}). 
	\item Gregor Hohpe and Bobby Woolf: Enterprise Integration Patterns\footnote{\url{http://www.eaipatterns.com/}} - this site provides a consistent vocabulary and visual notation to describe large-scale integration solutions across many implementation technologies. It also explores in detail the advantages and limitations of asynchronous messaging architectures. 
\end{itemize}

But both initiatives don't touch the intersection of software architecture and system administration. Therefore we want to introduce a set of patterns which bridges this gap. The problems that are cited in the aforementioned article have been experienced within daily system administration practice. With these patterns we want to give a handle to application developers to improve their applications from a system administration viewpoint.
\subsection*{Further Ideas} 
\begin{itemize}
	\item Hergebruik Zorg voor een standaardbibliotheek van standaard gebruikte oplossingen. Bijv. Logging, Security, Datumafhandeling, format van veelgebruikte user-data. Beheerders ondervinden veel last van dat voor iedere applicatie dezelfde functionaliteit toch net even anders gerealiseerd is en anders geconfigureerd. \textit{Comment Christian: dit zou in een pattern gaan richting {\sc Standardized Administration Tools} ofzo, dit zou dan wel {\sc Use Built-in System Logging} kunnen vervangen of specifieker maken.} 
	\item Goede functionele beschrijving. Redelijke beschrijving van wat een systeem globaal doet, uitgebreid met technische details rondom de werking. Waar draait de applicatie, waarom is deze applicatie nodig? Welke interfaces zijn er, wie gebruikt deze interfaces? Configuratie instellingen. \textit{Comment Christian: dit staat ook in het CACM artikel. Maar waarom is dit belangrijk, dus wat is het probleem als je het niet doet?} 
	\item NFR`s Hoeveel verkeer, hoe lang mag een transactie duren? Welke patterns worden gebruikt: synchroon/asynchroon. Heldere QoS/SLA: Beschikbaarheid, support, onderhoudsvensters, audit en logging. \textit{Comment Christian: Hier weet ik nog niet in hoeverre dat binnen de scope van dit paper valt. Kunnen we morgen bespreken.} 
\end{itemize}
