\section{The Patterns}

%Leo: 1st paragraph to intro
%Leo: Separate sections per pattern?

There have been several initiatives to describe patterns from the perspective of a system administrator, but these are mainly focused on infrastructure and middleware. Examples of these initiatives are: 
%Rebecca: Belong somewhere else or less verbose (Roland)
\begin{itemize}
	\item Daniel Jumelet: Open Infrastructure Architecture repository (OIAr) \cite{OIAm2014} - this site provides a wide variety of infrastructure patterns for several working areas: Client Realm, Middleware, Network, Security + Support, Server, Storage. Beside this repository also contains architecture \& design guidelines in the form of construction models at various levels and from various angles. It is constructed by making use of one of the most important tools of OIAm: The Building Blocks Model. 
	%R:Not meant for an intro: The Building Blocks Model is primarily a \textit{decomposition} tool. That means that it is used to dissect infrastructure landscapes into logical dimensions and parts in order to enable structured and methodological modeling (\textit{composition}). 
	\item Gregor Hohpe and Bobby Woolf: Enterprise Integration Patterns\cite{Hohpe2003} - this book provides a consistent vocabulary and visual notation to describe large-scale integration solutions across many implementation technologies. 
	%R:Not meant for an intro: It also explores in detail the advantages and limitations of asynchronous messaging architectures. 
\end{itemize}

In this paper we present three software architecture patterns for system administration support: 
%Unknown1: The origin of these patterns? Mainly if they are specialisations of exising ones?
\begin{itemize} 
	\item {\sc Administration API}
	\item {\sc Single File Location and Structure}
	\item {\sc Centralized System Logging}
%	\item {\sc Centralized Identity Management}
%	\item {\sc Multi-Tenant Application}
\end{itemize}

The patterns use an adapted version of the Alexandrian pattern format, as described in \cite{alexander1977}. The first part of each pattern is a short description of the context, followed by three diamonds. In the second part, the problem (in bold) and the forces are described, followed by another three diamonds. The third part offers the solution (again in bold), consequences of the pattern application --- which are part of the resulting context --- and a discussion of possible implementations. In the final part of each pattern, shown in \textit{italics}, we discuss related patterns, offer a rationale for the pattern based on literature and present know uses.
