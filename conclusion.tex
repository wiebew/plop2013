\section{Conclusion} 

In the article \textit{A plea from sysadmins to software vendors: 10 Do's and Don'ts} by Thomas Limoncelli \cite{Limoncelli2011a}, system administrators collected a basic list of do's and dont's for software vendors in order to make the life of the system administrators more easy. 

For a large number of points in this list there is a high agreement between administrators on what the best practices should be. However as system adminstrators are on the "receiving" end for a new or modified application, it is necessary to influence other parties who have a key position in the creation or the changing of an application. 

A role that fits this key position is the software architect. Among other concerns the software architect is responsible for the software architecture. The software architecture is the main design document for the software of an application. The design decisions taken in that document have a profound impact on the workload of the system administrators. 

This paper aims at influencing the software architects and the software architecture by providing patterns for software architecture that are endorsed by system administrators.

There have been several initiatives to describe patterns from the perspective of a system administrator, but these are mainly focused on infrastructure and middleware. Examples of these initiatives are: 
\begin{itemize}
	\item Open Infrastructure Architecture repository (OIAr)\footnote{\url{http://www.infra-repository.org/oiar/index.php/Main_Page}} 
	\item Enterprise Integration Patterns\footnote{\url{http://www.eaipatterns.com/}}
\end{itemize}

Both approaches --- software architecture patterns for realizing the above described quality attributes and patterns that support the work of system administrators --- don't touch some important aspects of the intersection of software architecture and system administration. Therefore we want to introduce a set of patterns which bridges this gap, based on the needs of the system administrators. 

The problems that are cited in the aforementioned article have been experienced within daily system administration practice. 

With this starting point for a repository of this kind of patterns we want to give ideas to software architects and application developers on how to improve their applications from a system administration viewpoint. Beside these patterns we want to bridge the gap between system administrators and the software architects of the software which needs to be administered by these system administrators.

